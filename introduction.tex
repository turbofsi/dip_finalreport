\section{Introduction}
Neural networks have received increasing usage in the areas of pattern recognition. One of the important applications is in character recognition. Character recognition has many practical interests, such as zip code recognition, document analysis. Many questions are usually involved, such as the choice of data representation, the design of classification algorithm and the selection of training data. In using neural net classifier, many researchers were choosing the multi-layer feed forward model with back-propagation algorithms. In this report, we choose to investigate the handwritten character(isolated digits) recognition project by using the discrete Hopfield network model as a pattern classifier. The data representation is based on image pixels. This type of representation is simple and straightforward. \\

Neural networks process information in a similar way the human brain does. The network is composed of a large number of highly interconnected processing elements working in parallel to solve a specific problem. Neural networks learn by example. A neuron has many inputs and one output. The neuron has two modes of operation:
\begin{enumerate}[label=(\alph*)]
\item Training mode. In the training mode, the neuron can be trained for particular input patterns.
\item Using mode. In the using mode, when a taught input pattern is detected at the input, its associated output becomes the current output.
\end{enumerate} 
If the input pattern does not belong in the taught list of input patterns, the training rule is used. Neural network has many applications. The most likely applications for the neural networks are classification, association and reasoning. An important application of neural networks is pattern recognition. Pattern recognition can be implemented by using a feed-forward neural network that has been trained accordingly. During training, the network is trained to associate outputs with input patterns. When the network is used, it identifies the input pattern and tries to output the associated output pattern.